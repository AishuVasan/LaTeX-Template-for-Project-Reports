\documentclass[12pt]{article}
\usepackage{longtable}
\begin{document}
\begin{center}
\begin{longtable}{|p{3cm}|p{3cm}|p{3cm}|p{3cm}|p{2.1cm}|}

\hline
\textbf{Raw materials} & \textbf{Procedure} & \textbf{Advantages}& \textbf{Disadvantages}&\textbf{References}\\
\hline
\endfirsthead
\multicolumn{4}{c}%
{\tablename\ \thetable\ -- \textit{Continued from previous page}} \\
 \hline
\textbf{Raw materials} & \textbf{Procedure} & \textbf{Advantages}& \textbf{Disadvantages}&\textbf{References}\\
\hline
\endhead
\hline \multicolumn{5}{r}{\textit{Continued on next page}} \\
\endfoot
\hline
\endlastfoot
 % row 1
Styrene & Styrene is burnt in limited supply of air. Carbon soot is collected and stirred at 900 rpm with NaOH (pH 7.4) for 25 mins and sonicated and filtered in whatmann filter paper &

\begin{itemize} 
	\item{Easier to prepare} 
	\item{Carbon dots are hydrophobic}
\end{itemize} 
& 
\begin{itemize} 
	\item{Extreme Reaction conditions} 
	\item{Carbon dots formed are of not uniform size}
\end{itemize} 
& \cite{mitra2012rapid}\\
\hline

%row 2
Hexadecylamine, Octadecene and Citric acid &
15ml Octadecene and 1.5g of HAD heated to 300\si{\degree}C in argon atmosphere and 1g CA is added and reaction is carried out for 30 mins. End product was purified using acetone	&
\begin{itemize}
	\item{CDs are hydrophobic}
	\item{CDs are of uniform size}
	\item{High quantum yield}
\end{itemize}
&
\begin{itemize}
	\item{Process is complex}
\end{itemize}
&
\cite{wang2010one}\cite{panniello2017luminescent}
\\
\hline
%row 3
Polyurethane &
Water borne polyurethane emulsion was  prepared and it is heated to 200°C for 24 hours with heating rate of 5\si{\degree}C/min. The mixture is cooled down and both water soluble as well as oil soluble CDs are obtained &
\begin{itemize}
	\item {Both water soluble as well as oil soluble CDs obtained}
\end{itemize} &
\begin{itemize}
	\item {Process is tedious}
\end{itemize} &
\cite{gu2016one}\\
\hline
%row 4
Pheophytin &
Pheophytin powder is added to N, N-dimethylformamide and ultra-sonicated. The mixture is heated at 150°C for 30 mins. Cooling down the sample and filtering will yield CDs & 
\begin{itemize} 
	\item{Solubility of the CDs can be altered}
	\item{CDs prepared is useful for Bio-imaging}
\end{itemize}&
\begin{itemize}
	\item {CDs are not of same size}
\end{itemize}&
\cite{wen2019pheophytin}\\
\hline

\end{longtable}
\end{center}
\end{document}